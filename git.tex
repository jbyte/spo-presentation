%%%%%%%%%%%%%%%%%%%%%%%%%%%%%%%%%%%%%%%%%
% Beamer Presentation
% LaTeX Template
% Version 1.0 (10/11/12)
%
% This template has been downloaded from:
% http://www.LaTeXTemplates.com
%
% License:
% CC BY-NC-SA 3.0 (http://creativecommons.org/licenses/by-nc-sa/3.0/)
%
%%%%%%%%%%%%%%%%%%%%%%%%%%%%%%%%%%%%%%%%%

%----------------------------------------------------------------------------------------
%	PACKAGES AND THEMES
%----------------------------------------------------------------------------------------

\documentclass{beamer}

\mode<presentation> {

% The Beamer class comes with a number of default slide themes
% which change the colors and layouts of slides. Below this is a list
% of all the themes, uncomment each in turn to see what they look like.

% \usetheme{default}
% \usetheme{AnnArbor}
% \usetheme{Antibes}
% \usetheme{Bergen}
\usetheme{Berkeley}
% \usetheme{Berlin}
% \usetheme{Boadilla}
% \usetheme{CambridgeUS}
% \usetheme{Copenhagen}
% \usetheme{Darmstadt}
% \usetheme{Dresden}
% \usetheme{Frankfurt}
% \usetheme{Goettingen}
% \usetheme{Hannover}
% \usetheme{Ilmenau}
% \usetheme{JuanLesPins}
% \usetheme{Luebeck}
% \usetheme{Madrid}
% \usetheme{Malmoe}
% \usetheme{Marburg}
% \usetheme{Montpellier}
% \usetheme{PaloAlto}
% \usetheme{Pittsburgh}
% \usetheme{Rochester}
% \usetheme{Singapore}
% \usetheme{Szeged}
% \usetheme{Warsaw}

% As well as themes, the Beamer class has a number of color themes
% for any slide theme. Uncomment each of these in turn to see how it
% changes the colors of your current slide theme.

% \usecolortheme{albatross}
% \usecolortheme{beaver}
% \usecolortheme{beetle}
% \usecolortheme{crane}
% \usecolortheme{dolphin}
% \usecolortheme{dove}
% \usecolortheme{fly}
% \usecolortheme{lily}
% \usecolortheme{orchid}
% \usecolortheme{rose}
% \usecolortheme{seagull}
% \usecolortheme{seahorse}
\usecolortheme{whale}
% \usecolortheme{wolverine}

%\setbeamertemplate{footline} % To remove the footer line in all slides uncomment this line
%\setbeamertemplate{footline}[page number] % To replace the footer line in all slides with a simple slide count uncomment this line

%\setbeamertemplate{navigation symbols}{} % To remove the navigation symbols from the bottom of all slides uncomment this line
}

\usepackage{graphicx} % Allows including images
\usepackage{booktabs} % Allows the use of \toprule, \midrule and \bottomrule in tables
\usepackage{listings}
\usepackage[utf8x]{inputenc}   % omogoča uporabo slovenskih črk kodiranih v formatu UTF-8
\usepackage[slovene]{babel}    % naloži, med drugim, slovenske delilne vzorce

%----------------------------------------------------------------------------------------
%	TITLE PAGE
%----------------------------------------------------------------------------------------

\title[Git]{Sistemska programska oprema\\Git} % The short title appears at the bottom of every slide, the full title is only on the title page

\author{Jaka Vute} % Your name
\institute[UL FRI] % Your institution as it will appear on the bottom of every slide, may be shorthand to save space
{
  Univerza v Ljubljani \\
  Fakulteta za računalništvo in informatiko \\ % Your institution for the title page
\medskip
\textit{jv5542@student.uni-lj.si} % Your email address
}
\date{\today} % Date, can be changed to a custom date

\begin{document}

\begin{frame}
\titlepage % Print the title page as the first slide
\end{frame}

\begin{frame}
\frametitle{Vsebina} % Table of contents slide, comment this block out to remove it
\tableofcontents % Throughout your presentation, if you choose to use \section{} and \subsection{} commands, these will automatically be printed on this slide as an overview of your presentation
\end{frame}

%----------------------------------------------------------------------------------------
%	PRESENTATION SLIDES
%----------------------------------------------------------------------------------------

%------------------------------------------------
\section{Struktura}
%------------------------------------------------
\subsection{Struktura imenika}
\begin{frame}[shrink=28]
  \frametitle{Struktura imenika}
  \begin{columns}[c]
    \begin{column}{.5\textwidth}
      \includegraphics[width=\textwidth]{pics/tree.png}
    \end{column}
    \begin{column}{.5\textwidth}
      \begin{itemize}
        \item branches(direktorij): hrani krajšave za URL-je, ki se uporabljajo
          pri git fetch, git pull, git push, ... (zastarelo)
        \item config(datoteka): konfiguracijska datoteka za specifičen projekt
        \item description(datoteka): uporablja samo GitWeb za prikaz opisa projekta
        \item HEAD(datoteka): hrani referenco do trenutne \emph{veje}, se
          uporabi kot starš naslednjega \emph{commit-a}
        \item hooks(direktorij): vsebuje primere skript, ki se lahko izvajajo
          pred ali po izvedbi določenega ukaza
        \item info/exclude(datoteka): podobno kot \emph{.gitignore} vendar ni verizonirana
        \item objects(direktorij): vsebuje podatke of git objektih (vsebina
          datotek, commit-i, tag-i)
        \item refs(direktorij): hrani reference na commit-e, tag-e in veje
      \end{itemize}
    \end{column}
  \end{columns}
\end{frame}

\subsection{Podatkovne strukture}
\begin{frame}[shrink=12]
  \frametitle{Podatkovne strukture}
  \begin{block}{Tipi podatkovnih struktur}
    \begin{itemize}
      \item \emph{index}: spremenjliv predpomnilnik, ki hrani informacije o trenutnem
        delovnem direktoriju in naslednji reviziji (.git/index)
      \item \emph{podatkovna baza objektov}: datoteke, ki se lahko samo dodajajo
        ne pa brišejo (.git/objects/)
    \end{itemize}
  \end{block}
  \begin{block}{Tipi objektov}
    \begin{itemize}
      \item blob: \emph{binary large object} vsebuje vsebino datoteke
      \item tree: ekvivalentno direktoriju, vsebuje seznam imen datotek vsaka s
        tipom in referenco na \emph{blob} ali drug \emph{tree} objekt
      \item commit: poveže \emph{tree} objekte v zgodovino, vsebuje ime
        \emph{tree} objekta, časovni žig in nič ali več imen starševskih
        \emph{commit} objektov
      \item tag: vsebuje referenco na nek drug objekt in lahko hrani
        meta-podatke povezane na drug objekt
    \end{itemize}
  \end{block}
\end{frame}

%------------------------------------------------
\section{Delovanje ukazov}
%------------------------------------------------
\subsection{Git init}
\begin{frame}
  \frametitle{Git init}
  \begin{columns}[c]
    \begin{column}{.5\textwidth}
      \begin{block}{Git init (lokalno)}
        \includegraphics[width=\textwidth]{pics/init.png}
      \end{block}
    \end{column}
    \begin{column}{.5\textwidth}
      \begin{block}{Git init --bare (oddaljeno)}
        \includegraphics[width=\textwidth]{pics/init-bare.png}
      \end{block}
    \end{column}
  \end{columns}
\end{frame}

\subsection{Git add}
\begin{frame}
  \frametitle{Git add}
  \begin{columns}[c]
    \begin{column}{.5\textwidth}
      \emph{\$ touch test.txt}\\
      \emph{\$ mkdir folder}\\
      \emph{\$ touch folder/test2.txt}\\
      \emph{\$ echo 'test line' $>$ test.txt}\\
      \emph{\$ echo 'test line 2' $>$ folder/test2.txt}\\
      \emph{\$ git add .}
    \end{column}
    \begin{column}{.5\textwidth}
      Ustvari nova \emph{blob} objekta:
      \includegraphics[width=\textwidth]{pics/add-tree.png}
      Ki vsebujeta vsebino dodanih datotek:
      \includegraphics[width=\textwidth]{pics/add-content.png}
      V \emph{.git/index} se zapišejo dovoljenja, ime blob objekta in pot do
      dodane datoteke:
      \includegraphics[width=\textwidth]{pics/add-index.png}
    \end{column}
  \end{columns}
\end{frame}

\subsection{Git commit}
\begin{frame}
  \frametitle{Git commit}
  \emph{\$ git commit -m 'Add test files'}
  \begin{columns}[c]
    \begin{column}{.5\textwidth}
      \begin{enumerate}
        \item Nova datoteka: \emph{.git/COMMIT\_EDITMSG}
        \item Novi objekti: \emph{9353cde3, cf3d1dc4, 9c9f2d67}
        \item Nova datoteka: \emph{.git/refs/heads/master}
        \item Nova datoteka: \emph{.git/logs/refs/heads/master}
      \end{enumerate}
    \end{column}
    \begin{column}{.5\textwidth}
      Vsebina .git/COMMIT\_EDITMSG:
      \includegraphics[width=\textwidth]{pics/commit-msg.png}
      Vsebina .git/objects/:
      \includegraphics[width=\textwidth]{pics/commit-tree.png}
      Vsebina .git/refs/heads/master:
      \includegraphics[width=\textwidth]{pics/commit-refs.png}
    \end{column}
  \end{columns}
  Vsebina .git/logs/refs/heads/master:
  \includegraphics[width=\textwidth]{pics/commit-log.png}
\end{frame}

\begin{frame}
  \frametitle{Git commit (nadaljevanje)}
  Vsebina \emph{tree} objektov:
  \includegraphics[width=\textwidth]{pics/commit-content-tree.png}
  Vsebina \emph{commit} objektov:
  \includegraphics[width=\textwidth]{pics/commit-content-commit.png}
\end{frame}

\subsection{Git branch}
\begin{frame}
  \frametitle{Git branch}
  \emph{\$ git branch feature}
  \newline
  Novi datoteki: \emph{.git/refs/heads/feature} in \emph{.git/logs/refs/heads/feature}
  \includegraphics[width=\textwidth]{pics/branch-refs.png}
  \includegraphics[width=\textwidth]{pics/branch-logs.png}
\end{frame}

\subsection{Git push}
\begin{frame}
  \frametitle{Git push (lokalno)}
  \emph{\$ git remote add origin $\sim$/projects/tmp/git-remote}\\
  \emph{\$ git push -u origin master}
  \newline
  \begin{columns}[c]
    \begin{column}{.5\textwidth}
      V \emph{.git/config} se zapišejo podatki o oddaljeni veji:
      \includegraphics[width=\textwidth]{pics/push-remote.png}
      \includegraphics[width=\textwidth]{pics/push-branch.png}
    \end{column}
    \begin{column}{.5\textwidth}
      Novi datoteki:
      \includegraphics[width=.8\textwidth]{pics/push-tree.png}
    \end{column}
  \end{columns}
\end{frame}

\begin{frame}
  \frametitle{Git push (oddaljeno)}
  \includegraphics[width=.5\textwidth]{pics/push-remote-tree.png}
\end{frame}

\subsection{Git fetch}
\begin{frame}[shrink=20]
  \frametitle{Git fetch (priprava)}
  \emph{\$ git clone $\sim$/projects/tmp/git-remote $\sim$/projects/tmp/git-clone}\\
  \emph{\$ echo 'test fetch' $>>$ test.txt}\\
  \emph{\$ git add .}\\
  \emph{\$ git commit -m 'Add new line to test.txt'}\\
  \emph{\$ git push origin master}
  \newline
  \begin{columns}[c]
    \begin{column}{.5\textwidth}
      \includegraphics[width=\textwidth]{pics/fetch-prep.png}
    \end{column}
    \begin{column}{.5\textwidth}
      \includegraphics[width=\textwidth]{pics/fetch-prep-blob.png}
    \end{column}
  \end{columns}
\end{frame}

\begin{frame}
  \frametitle{Git fetch}
  \emph{\$ git fetch}
  \newline
  \begin{columns}[c]
    \begin{column}{.5\textwidth}
      Prenesejo se novi objekti z oddaljenega skladišča:
      \includegraphics[width=\textwidth]{pics/fetch-prep.png}
    \end{column}
    \begin{column}{.5\textwidth}
      Datoteka \emph{.git/refs/remotes/origin/master} vsebuje referenco na
      najnovejši \emph{commit}:
      \includegraphics[width=\textwidth]{pics/fetch-refs.png}
    \end{column}
  \end{columns}
  Ustvari se tudi datoteka \emph{.git/FETCH\_HEAD}:
  \includegraphics[width=\textwidth]{pics/fetch-head.png}
\end{frame}

\subsection{Git merge}
\begin{frame}
  \frametitle{Git merge (fast-forward)}
  \emph{\$ git merge origin/master}
  \newline
  Preprosto nastavi \emph{.git/refs/heads/master} na isti \emph{commit}, kot je
  v \emph{.git/refs/remotes/origin/master}:
  \includegraphics[width=\textwidth]{pics/merge-ff.png}
\end{frame}

\begin{frame}[shrink=20]
  \frametitle{Git merge (--no-ff)}
  \emph{git-clone \$ echo 'test merge --no-ff' $>>$ test.txt}\\
  \emph{git-clone \$ git add . \&\& git commit -m 'Add line to test merge --no-ff'}\\
  \emph{git-clone \$ git push origin master}\\
  \emph{git \$ git fetch}\\
  \emph{git \$ git merge --no-ff origin/master}
  \newline
  \includegraphics[width=.4\textwidth]{pics/merge-no-ff.png}
\end{frame}

%------------------------------------------------
\section{Viri}
%------------------------------------------------

\begin{frame}
\frametitle{Viri}
\footnotesize{
\begin{thebibliography}{99} % Beamer does not support BibTeX so references must be inserted manually as below
  \bibitem{sggit} SiteGround
    \newblock SiteGround Git Directory Structure Tutorial
    \newblock \url{https://www.siteground.com/tutorials/git/directory-structure}
    \newblock [Dostopano: 30.12.2017]

  \bibitem{gitscm} git-scm
    \newblock git-scm documentation
    \newblock \url{https://git-scm.com/doc}
    \newblock [Dostopano: 30.12.2017]

  \bibitem{gitwiki} Wikipedia
    \newblock Git
    \newblock \url{https://en.wikipedia.org/wiki/Git}
    \newblock [Dostopano: 30.12.2017]
\end{thebibliography}
}
\end{frame}

%------------------------------------------------

% \begin{frame}
% \frametitle{Bullet Points}
% \begin{itemize}
% \item Lorem ipsum dolor sit amet, consectetur adipiscing elit
% \item Aliquam blandit faucibus nisi, sit amet dapibus enim tempus eu
% \item Nulla commodo, erat quis gravida posuere, elit lacus lobortis est, quis porttitor odio mauris at libero
% \item Nam cursus est eget velit posuere pellentesque
% \item Vestibulum faucibus velit a augue condimentum quis convallis nulla gravida
% \end{itemize}
% \end{frame}

%------------------------------------------------

% \begin{frame}
% \frametitle{Blocks of Highlighted Text}
% \begin{block}{Block 1}
% Lorem ipsum dolor sit amet, consectetur adipiscing elit. Integer lectus nisl, ultricies in feugiat rutrum, porttitor sit amet augue. Aliquam ut tortor mauris. Sed volutpat ante purus, quis accumsan dolor.
% \end{block}

% \begin{block}{Block 2}
% Pellentesque sed tellus purus. Class aptent taciti sociosqu ad litora torquent per conubia nostra, per inceptos himenaeos. Vestibulum quis magna at risus dictum tempor eu vitae velit.
% \end{block}

% \begin{block}{Block 3}
% Suspendisse tincidunt sagittis gravida. Curabitur condimentum, enim sed venenatis rutrum, ipsum neque consectetur orci, sed blandit justo nisi ac lacus.
% \end{block}
% \end{frame}

%------------------------------------------------

% \begin{frame}
% \frametitle{Multiple Columns}
% \begin{columns}[c] % The "c" option specifies centered vertical alignment while the "t" option is used for top vertical alignment

% \column{.45\textwidth} % Left column and width
% \textbf{Heading}
% \begin{enumerate}
% \item Statement
% \item Explanation
% \item Example
% \end{enumerate}

% \column{.5\textwidth} % Right column and width
% Lorem ipsum dolor sit amet, consectetur adipiscing elit. Integer lectus nisl, ultricies in feugiat rutrum, porttitor sit amet augue. Aliquam ut tortor mauris. Sed volutpat ante purus, quis accumsan dolor.

% \end{columns}
% \end{frame}

%------------------------------------------------

% \begin{frame}
% \frametitle{Table}
% \begin{table}
% \begin{tabular}{l l l}
% \toprule
% \textbf{Treatments} & \textbf{Response 1} & \textbf{Response 2}\\
% \midrule
% Treatment 1 & 0.0003262 & 0.562 \\
% Treatment 2 & 0.0015681 & 0.910 \\
% Treatment 3 & 0.0009271 & 0.296 \\
% \bottomrule
% \end{tabular}
% \caption{Table caption}
% \end{table}
% \end{frame}

%------------------------------------------------

% \begin{frame}
% \frametitle{Theorem}
% \begin{theorem}[Mass--energy equivalence]
% $E = mc^2$
% \end{theorem}
% \end{frame}

%------------------------------------------------

% \begin{frame}[fragile] % Need to use the fragile option when verbatim is used in the slide
% \frametitle{Verbatim}
% \begin{example}[Theorem Slide Code]
% \begin{verbatim}
% \begin{frame}
% \frametitle{Theorem}
% \begin{theorem}[Mass--energy equivalence]
% $E = mc^2$
% \end{theorem}
% \end{frame}\end{verbatim}
% \end{example}
% \end{frame}

%------------------------------------------------

% \begin{frame}
% \frametitle{Figure}
% Uncomment the code on this slide to include your own image from the same directory as the template .TeX file.
% \begin{figure}
% \includegraphics[width=0.8\linewidth]{test}
% \end{figure}
% \end{frame}

%------------------------------------------------

% \begin{frame}[fragile] % Need to use the fragile option when verbatim is used in the slide
% \frametitle{Citation}
% An example of the \verb|\cite| command to cite within the presentation:\\~

% This statement requires citation \cite{p1}.
% \end{frame}


\end{document} 
